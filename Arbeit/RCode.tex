\section{R-Code}
\label{sec:R-Code}

We conducted our analysis in the statistical software R \cite{R:2022}. For our data cleansing, data handling and plotting we use the \textit{tidyverse} package \cite{Tidyverse:2019}. Further we use the packages \textit{here} \cite{here:2020}, \textit{miceadds} \cite{Miceadds:2023} and \textit{parallel}, which is part of core R, to facilitate our analysis.  

For building our CoDA model we use the packages \textit{vars} \cite{VAR:2008,CoDAR2:2008} and \textit{robCompositions} \cite{RobComp:2011,CoDAR4:2018}. Especially the functions \texttt{pivotCoord}, which performs the ilr transformation described in \ref{sec: Coda Preliminaries},\textit{VAR}, which builds the VAR model described in \ref{sec:The VAR Model}, and \texttt{D2invPC} which performs the necessary back transformation to get predictions in the desired space. The INGARCH(p,q) analysis is mainly done with the package \textit{tscount} \cite{Tscount:2017,Tscount:2020}. The core function used is \texttt{tsglm} which we use to fit the INGARCH(p,q) model as well as the log-linear model. The zero-inflated model were fitted using the function \texttt{zeroinfl} from the package \textit{pscl} \cite{Pscl:2008}. For the VGAM we used the package \textit{VGAM} \cite{RVGAM:2010}. 

Since we focus our efforts on the INGARCH(p,q) and CoDA model, we will only describe the functions used by them. In general, all functions can be grouped into three categories: general, INGARCH specific and CoDA specific. General functions are used for both, the INGARCH(p,q) and the CoDA model. INGARCH and CoDA specific functions are only used for their respective methods. 

Notable general functions are \texttt{Data.Window} and \texttt{Data.Preparation}. The former function splits the time series in the specified windows and the value to be predicted. The models are then fitted on these windows and the prediction result is compared with the actual value. The latter one brings the data in the right format, replaces missing values with 0 and accounts for the length of the history chosen. In addition, for CoDA it also transform the data into the right format needed for the one-vs-all method. Other important functions are \texttt{Model.Error} and \texttt{Model.ErrorOverall} which implement the error measure introduced in \ref{sec: Error Measure} and summarises it. 

There are three INGARCH specific functions. The first function is \texttt{Ingarch.DataPreparation} which transforms the data into the right format needed to fit the INGARCH(p,q) model. At its core it uses the \texttt{Data.Preparation} function but adds the additional option to replace zero values with 1. The second function is \texttt{Ingarch.Prediction} which is the function where the model is fit and the predicted value is calculated. It used the \texttt{tsglm} function to fit the model for each window and predicts the next value. The third function is \texttt{Ingarch.Analysis} which acts as a wrapper function to streamline and facilitate the analysis. The previously mentioned model specifications can be chosen here as well as various other options. 

The CoDA specific functions have the same structure as the INGARCH ones. Again there are \texttt{Coda.DataPreparation},\texttt{Coda.Prediction} and \texttt{Coda.Analysis} and they act like their respective INGARCH counterparts. \texttt{Coda.DataPreparation} transforms the data into the correct format, \texttt{Coda.Prediction} fits the model, predicts the future value and compares it with the true value and \texttt{Coda.Analysis} is again the wrapper function where the mentioned specifications can be chosen. 