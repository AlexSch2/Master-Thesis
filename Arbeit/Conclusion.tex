In this thesis, we compare multiple models for multivariate count data with an excessive amount of zeros. A special focus lies in the integer-valued generalized autoregressive conditional heteroskedasticity model of order (p,q) (INGARCH(p,q) model) and the compositional data analysis (CoDA) model. The other models include a zero-inflated model (ZIM) and an integer-valued autoregressive model of order p (INAR(p) model). Since this thesis is carried out as part of a bigger project at the Technical University of Austria in cooperation with Schrankerl GmbH, we were able to test our models on real world data and compare them with the model currently in use. 

The current model employed by the company is a naive random walk model. While in general, all tested models outperform the current model and deliver a similar output, they come with their respective advantages and disadvantages.

The INGARCH model considers the discrete nature of our data, but ignores the multivariate structure of our data and the appearance of an unusual amount of zeros. While there exists a multivariate version of the INGARCH model, to our knowledge, there is no software implementation in R. In addition, the INGARCH(p,q) model assumes the data to be Poisson or Negative Binomial distributed and hence does not accommodate to the excessive amount of zeros. The fitting of a multivariate INGARCH model and assuming the data to be ZIP distributed, can be grounds for further investigations. 

The CoDA model fits a multivariate model and sees our data as a compositional time series. While it also neglects the fact that we have integer-valued data, the biggest issue with CoDa is, that the data must not include zeros. While there exist various methods to handle zero values as presented in Section \ref{sec: Zero-Handling}, the handling of essential zeros, which is what we have, is especially tricky. We opted for the suggested data amalgamation and the simple replacement strategy because of their simplicity. A better way of handling zeros could be worth future research. 

The INAR model has similar assumptions as the INGARCH model. It accounts for the discrete nature of our data, but neglects its multivariate structure and the amount of excessive zeros. The excessive zeros seem to be troublesome for the INAR model in general, since it performs the worst out of all models for time series with many zeros. 

The zero-inflated model is an intriguing model. It considers both, the excessive amounts of zeros and the discrete nature of our data into account and only ignores the multivariate structure of it. However, there is one major drawback to it. While we do have an excessive amount of zeros, we do not necessarily have zeros in every time series and especially not if we only use parts of them. But since the zero-inflated model needs to have at least one zero in the sample, it cannot be fit on samples with no zeros present. This restricts us to fit the ZIM only on the categories with the most zeros present. 

While we did some small tuning and compared different parameter settings for our models, we did not conduct an extensive analysis. Using some common model selection criteria like the Akaike information criteria (AIC) or Bayesian information criteria (BIC) could be interesting for further work. Other possible future work could be the analysis of time series specific characteristics like seasonality and stationarity. Especially the test for seasonality could be interesting since many offices are emptier during holiday season and hence have less possible costumers than usual. 

Although we tried out many different models and conducted a literature review, it turns out to be difficult to find a model which takes the three major characteristics of our data, integer-valued, multivariate and excessive amount of zeros, into account. While all the models outperform the naive random walk model, it would still be interesting to find a model which takes all three major characteristics of our data into account. In addition, the development of a multivariate INGARCH software package or the theoretical extension of ZIM to multivariate data are interesting points for future research. 