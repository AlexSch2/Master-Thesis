In this thesis, we analyse and compare two approaches for multivariate count data time series with an excessive amount of zeros. The first approach belongs to the class of generalised linear models (GLM) and fits a univariate integer-valued generalized autoregressive conditional heteroskedasticity model of order (p,q) (INGARCH(p,q) model) for each dimension. The second approach is based on compositional data analysis (CoDA) and uses the relative structure of our data to build a vectorised autoregressive (VAR) model from it. In addition, we also consider alternative options like zero-inflated models (ZIM) and integer-valued autoregressive (INAR) models. Providing the mathematical background for the INGARCH(p,q) and CoDA approach and exploring different parameter settings for them, we evaluate their performance on real world data and compare different tuning options. We then introduce an error measure for comparison and use it to compare the performance on different time series. We provide a handbook of our analysis in the statistical software R and present the used packages and functions. At last, we show the results of our analysis. All models outperform the naive random walk model, but they cannot take all three major characteristics, integer-valued, multivariate structure and excessive amount of zeros, simultaneously into account.  \newline



\textbf{Keywords:} Compositional Data Analysis, General Linear Models, INGARCH, Multivariate Count Data Time Series, R