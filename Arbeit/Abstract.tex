In this thesis we analyse and compare two approaches for multivariate count data time series with an excessive amount of zeros. The first approach focuses on the class of generalised linear models (GLM). For each dimension of our data, it fits an univariate integer valued generalized autoregressive conditional heteroskedasticity model of order (p,q) (INGARCH(p,q) model). The second approach is based on compositional data analysis (CoDA). It uses the relative structure of our data and builds a vectorised autoregressive (VAR) model from there. In addition, we also consider other models like zero-inflated models or vectorised general additive models. We provide the mathematical background for the INGARCH(p,q) and CoDA approach and explore different specifications of them. We then test their performance on real world data and compare different tuning choices. We introduce an error measure to ease comparison and also use it to compare the performance on different time series. We conduct our analysis in the statistical software R and explain what packages were used and explain the most important functions.  