While multivariate count data time series with an excessive amount of zeros is a frequently encountered real-world problem, there is yet a clear way to handle them. This thesis is part of a bigger project carried out at the Technical University of Vienna in cooperation with the company Schrankerl GmbH. The company operates food vending machines in offices, which are restocked on a weekly basis. Since non-sold food gets disposed of, the company is in search of a model which can predict the amount sold in the upcoming week. For this, we compare various approaches with a focus on an integer-valued generalized autoregressive conditional heteroskedasticity model of order (p,q) (INGARCH(p,q) model) and a compositional data analysis (CoDA) model. Other investigated models include a zero-inflated model (ZIM) and an integer-valued autoregressive model of order p (INAR(p) model). In the first half of the thesis, we provide the mathematical background for these models. First, for the count time series models in Chapter \ref{sec:CountTS} and later, for the CoDA model in Chapter \ref{sec:Coda}. In the second half, we compare these models on a real-world data set provided to us by the company. This is done in Chapter \ref{sec:Application}. We investigate tuning options for our models in Section \ref{sec: Model Specification} and for comparison across different time series and the currently employed model, we introduce an error measure in Section \ref{sec: Error Measure}. We conduct our analysis in the statistical software R and provide a handbook for our code as well as an overview of all used packages. In Section \ref{sec:Results} we present the results of our analysis. Future extensions and possible further research options are mentioned in Chapter \ref{sec: Conclusion}. We have shown that our models outperform the currently employed model, but come with their respective advantages and disadvantages. Since, to our knowledge, there exists currently no model which takes all three characteristics, the multivariate structure, the integer nature, and the excessive amount of zeros, into account, the development of such a model poses an interesting basis for future research.

%our models also don't consider one or two of these characteristics
%While our models outperform the currently employed model, they each come with their respective advantages and disadvantages. They all fail to take into account either the multivariate structure of our data, the integer nature of it or the excessive amount of zeros. However, to our knowledge there is currently no model which takes all three characteristics into account. 
