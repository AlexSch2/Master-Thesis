While multivariate count data time series with an excessive amount of zeros is an often encountered real-world problem, there is yet not a clear way to handle them. This thesis is part of a bigger project carried out at the Technical University of Vienna in cooperation with the company Schrankerl GmbH. The company operates office food vending machines, which are restocked on a weekly basis. Since non-sold food gets disposed of, the company is in search of a model which can predict the amount sold in the upcoming week. For this, we compared various approaches with a focus on an integer-valued generalized autoregressive conditional heteroskedasticity model of order (p,q) (INGARCH(p,q) model), and a compositional data analysis (CoDA) model. Other investigated models include a zero-inflated model (ZIM) and a integer-valued autoregressive model of order p (INAR(p) model). In the first half of the thesis, we provided the mathematical background for these models. First, for the count time series models in Chapter \ref{sec:CountTS} and later, for the CoDA model in Chapter \ref{sec:Coda}. In the second half, we compared these models on a real-world data set provided to us by the company. This is done in Chapter \ref{sec:Application}. We investigated tuning options for our models in Section \ref{sec: Model Specification} and for comparison across different time series and the currently employed model, we introduced an error measure in Section \ref{sec: Error Measure}. We conducted our analysis in the statistical software R and provided a handbook for our code as well as an overview of all used packages. In Section \ref{sec:Results} we presented the results of our analysis.  While our models outperformed the currently employed model, there are still various further areas for future research, which are mentioned in Chapter \ref{sec: Conclusion}. 