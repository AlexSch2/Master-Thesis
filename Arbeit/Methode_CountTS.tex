\section{Motivation}
\label{sec:Ingarch Motivation}

In this section we introduce the different count time series models. We begin with a short literature review about possible count data models and then provide a motivation on why we decided to focus on our models. The review is mainly based on \cite{Liboschik:2016} and \cite{Heinen:2003} and a more detailed review can be found in \cite{Zucchini:1997}. Later, we define the models themselves and list some of their properties. 

Since our data can be seen as a discrete time series with count data, we want a model which is able to take these properties into account. Hence, common features of count data, like autocorrelation and over dispersion, should not be neglected and instead be modelled properly.
 
One common way to deal with count data are Markov chains. In Markov chains, the dependent variable can take on all possible values in the so called state space and the probability of changing states is then modelled as a transition probability. A limitation is the fact that these models become cumbersome if the state space gets too big and lose tractability. As an extension to the basic Markov chains models, Hidden Markov chains are proposed by \cite{Zucchini:1997}. However, since there is no generally accepted way to determine the order of this model, it can cause problems if the data structure does not provide intuitive ways to do it. Another issue is that the number of parameters which need to be estimated gets big quickly, especially if the order of the model is big. 

Other common models for time series data are the ARMA models and their discrete version, the Discrete Autoregressive Moving Average (DARMA) models. They can be defined as a mixture of discrete probability distributions and a suitable chosen marginal probability function \cite{Biswas:2009}. While there have been various applications, for example in \cite{Chang:1987}, there seem to be difficulties in their estimation \cite{Heinen:2003}. 

State space models with conjugated priors are proposed by \cite{Harvey:1989}.Here, one assumes that the observations are drawn from a Poisson distribution whose mean itself follows a Gamma distribution. The parameters of the Gamma distribution are chosen in such a way that its mean is constant but its variance is increasing. While there are ways proposed by \cite{Qaqish:1988} to handle overdispersion, these models have the weakness of needing further assumptions to handle zeros while also having more complicated model specifications \cite{Heinen:2003}.

We decide to focus on the class of Generalised Linear Models (GLM) and in particular on the INGARCH(p,q) and log-linear model. For those models, the observations are modelled conditionally on the past and follow a discrete distribution. The conditional mean is then connected with a link function to the past observations and conditional means. A covariate vector can be included in the model to factor in additional, external information. While being easy to use and estimate, they still provide a good amount of flexibility and additionally, a wide array of tools is available for various tests and forecasts. We also introduce an extension of the INGARCH model to multivariate data. However, since to our knowledge there is currently no R-package available to fit these models, we stay with the univariate version. The INGARCH(p,q) and log-linear model will be discussed in detail in \ref{sec:Ingarch} and \ref{sec: Log-Linear Models} respectively.

Since our data features many zero values, we also investigate zero-inflated models (ZIM) with the focus on a zero-inflated version of the INGARCH(p,q) model. The structure of this model follows an INGARCH(p,q) model, but with a zero-inflated Poisson distribution as the conditional distribution. However, due to a lack of appropriate R-packages, we use a slightly different version of the ZIM, which was introduced by \cite{Lambert:1992}. This model is basically a generalised linear regression model with a logit link where the data is assumed to follow a zero-inflated Poisson distribution. More details can be found in \ref{sec: Zim}

Another popular approach for count time series are the integer-valued autoregressive (INAR) models presented in section \ref{sec: Inar Distributional assumptions}. These models are based on a thinning operator and a parameter $\alpha$. The dependent variable $y_t$ is modelled as the sum of an error term and the sum of $y_{t-1}$ draws from a distribution with mean $\alpha$ and finite variance. They are attractive since they have a linear-like structure and a similar correlation structure to AR or ARMA models and hence can be seen as a discrete counterpart \cite{Heinen:2003}. 

The simple naive random walk in \ref{sec: Naive Random Walk} is the simplest and most basic approach not only for count data, but time series in general. This is the model that is currently used for forecasting and is therefore ideal as a benchmark. We will use it to compare the performance of the models with the help of a new error measure in \ref{sec: Error Measure}. 

Since the INGARCH and the INAR model are based on their real valued counterparts, the GARCH and AR model, we will also provide a short review for them for better comparison and clearness on why we choose the integer valued versions. However, we will not consider neither the GARCH nor the AR model in our analysis. 

\section{INGARCH Model}
\label{sec:Ingarch}

We construct the INGARCH(p,q) model as in \cite{Liboschik:2016}. 
Take again our time series $\left\{\bm{Y}_t:t=1,\ldots,T; \bm{Y}_t \in \mathbb{N}_0^K \right\}_f$ for fridge $f$ and denote the univariate time series for category $k$ with $\left\{Y_{kt}:t=1,\ldots,T; Y_{kt} \in \mathbb{N}_0\right\}_f$  for $k=1,\ldots,K$. This means $\bm{Y}_t = (Y_{1t},\ldots,Y_{Kt})^T$. Denote a r-dimensional time varying covariate vector with $\textbf{X}_{kt}=(X_{t1}^k,\ldots,X_{tr}^k)^T$. Let the conditional mean be $\lambda_{kt} = \mathbb{E}\left[Y_{kt} | \SigA_{k,t-1} \right]$ where $\SigA_{k,t-1}$ is the $\sigma$-field generated by $Y_{kt}$ and $\lambda_l$ for $l<t$ $, \SigA _{k,t-1}= \sigma(Y_{k1},\ldots,Y_{kl},\lambda_1, \ldots, \lambda_l)$. Therefore, the conditional mean of the time series is dependent on its combined history of the past conditional means and its past values. With this, we can define the integer valued generalized autoregressive conditional heteroskedasticity model of order (p,q) (INGARCH(p,q) model) for category $k=1,\ldots,K$ as,

\begin{gather}
\label{eq:Ingarch model}
Y_{kt} | \SigA_{k,t-1} \sim P(\lambda_{kt}); \forall t \in \mathbb{N}, \\
\mathbb{E}\left[Y_{kt} | \SigA_{k,t-1} \right] = \lambda_{kt} = \beta_0 + \sum_{i=1}^p\beta_i Y_{k,t-i} + \sum_{j=1}^q\alpha_j \lambda_{k,t-j},
\end{gather}

where $p,q \in \mathbb{N}$ and $P(\lambda_{kt})$ is a Poisson distribution with mean $\lambda_{kt}$. The integer $p$ defines the number of past values to regress on, whereas $q$ does the same for the past conditional means. In order to account for external effects as well, we add the covariate vector $\textbf{X}_{kt}$

\begin{gather}
\label{eq:Ingarch model with external effect}
Y_{kt} | \SigA_{k,t-1} \sim P(\lambda_{kt}); \forall t \in \mathbb{N}, \\
\mathbb{E}\left[Y_{kt} | \SigA_{k,t-1} \right] = \lambda_{kt} = \beta_0 + \sum_{i=1}^p\beta_i Y_{k,t-i} + \sum_{j=1}^q\alpha_j \lambda_{k,t-j} + \bm{\eta}^T\textbf{X}_{kt},
\end{gather}

where $\bm{\eta}$ is the parameter for the covariates such that $\bm{\eta}^T\textbf{X}_{kt} \geq 0$.
From the distributional assumption $Y_{kt} | \SigA_{k,t-1} \sim P(\lambda_{kt})$ it follows

\begin{equation}
p_{kt}(y;\bm{\theta})=\mathbb{P}(Y_{kt}=y | \SigA_{k,t-1}) = \frac{\lambda_{kt}^y \exp(-\lambda_{kt})}{y!}, \hspace{0.2cm} y \in \mathbb{N}_0.
\label{eq:Ingarch Distribution}
\end{equation}

Furthermore it can be shown that conditionally on the past history $\SigA_{k,t-1}$, the model is equidispersed, i.e. it holds $\lambda_{kt} = \mathbb{E}\left[Y_{kt} | \SigA_{k,t-1}\right] = \mathbb{V}\left[Y_{kt} | \SigA_{k,t-1}\right]$. However, unconditionally the model exhibits overdispersion. In that case it holds $\mathbb{E}\left[Y_{kt}\right] \leq \mathbb{V}\left[Y_{kt}\right] $ \cite{Heinen:2003}. 

\subsubsection{Parameter Estimation and Forecasting}
\label{sec: Estimation of the Ingarch Model}

We summarise the estimation of the INGARCH(p,q) Model as described in \cite{Liboschik:2016}. The model is estimated for each category $k=1,\ldots,K$ separately. 

The parameter space for the INGARCH(p,q) model with external effects \ref{eq:Ingarch model with external effect} is given by 

\begin{equation}
\Theta = \left\{ \bm{\theta} \in \mathbb{R}^{p+q+r+1}: \beta_0 > 0, \beta_1,\ldots,\beta_p,\alpha_1,\ldots,\alpha_q,\eta_1,\ldots,\eta_r \geq 0, \sum_{i=1}^p\beta_i + \sum_{j=1}^q\alpha_j < 1 \right\}.
\label{eq:Ingarch parameter space}
\end{equation}

To ensure positivity of the conditional mean $\lambda_{kt}$, the intercept $\beta_0$ must be positive while all other parameters must be non negative. The upper bound of the sum ensures that the model has a stationary and ergodic solution with moments of any order \cite{Ferland:2006,Fokianos:2009,Doukhan:2012}. A quasi maximum likelihood approach is used to estimate the parameters $\bm{\theta}$. 
For observations $\textbf{y}_k = \left(y_{k1},\ldots,y_{kT}\right)^T$ for category $k=1,\ldots,K$, the conditional quasi log-likelihood function, up to a constant, is given by,

\begin{equation}
\loglik_k(\bm{\theta}) = \sum_{t=1}^T\log p_{kt}(y_{kt};\bm{\theta}) = \sum_{t=1}^T \left(y_{kt}\log(\lambda_{kt}(\bm{\theta})) - \lambda_{kt}(\bm{\theta})\right).
\label{eq: Quasi log likelihood}
\end{equation}

where $p_{kt}(y_{kt};\bm{\theta})$ is the probability density function defined in \ref{eq:Ingarch Distribution}. The conditional mean is seen as a function $\lambda_{kt}: \Theta \rightarrow \mathbb{R}^{+}$. The conditional score function is given by,

\begin{equation}
S_{kT}(\bm{\theta}) = \frac{\partial \loglik_k(\bm{\theta})}{\partial \bm{\theta}} = \sum_{t=1}^T\left(\frac{y_{kt}}{\lambda_{kt}(\bm{\theta})}-1\right)\frac{\partial\lambda_{kt}(\bm{\theta})}{\partial \bm{\theta}}.
\label{eq:conditional score}
\end{equation}

The vector $\frac{\partial\lambda_{kt}(\bm{\theta})}{\partial \bm{\theta}}$ can be computed recursively. 
The conditional information matrix is given by, 

\begin{align}
G_{kT}(\theta) &= \sum_{t=1}^T Cov\left(\frac{\partial \loglik_k(\bm{\theta}; Y_{kt})}{\partial \bm{\theta}} \middle| \SigA_{k,t-1}\right) \\
&=  \sum_{t=1}^T \left(\frac{1}{\lambda_{kt}\left(\bm{\theta}\right)}\right) \left(\frac{\partial \lambda_{kt}(\bm{\theta})}{\partial \bm{\theta}}\right)\left(\frac{\partial \lambda_{kt}(\bm{\theta})}{\partial \bm{\theta}}\right)^T.
\label{eq:conditional information matrix}
\end{align}

Finally, assuming that the quasi maximum likelihood estimator (QMLE) $\hat{\bm{\theta}}_T$ of $\bm{\theta}$ exists, it is the solution to 

\begin{equation}
\hat{\bm{\theta}}:= \hat{\bm{\theta}}_T = \text{arg max}_{\bm{\theta} \in \Theta} (\loglik_k(\bm{\theta})). 
\label{eq:ingarch qmle}
\end{equation}

The optimal one-step ahead forecast with regards to the mean squared error is the conditional expectation $\lambda_{k,t+1}=\mathbb{E}\left[Y_{k,t+1} | \SigA_{kt} \right]$. The h-step ahead prediction for $h>1$ is calculated iteratively with the one step ahead predictions of $Y_{k,t+1},Y_{k,t+2},\ldots$ \cite{Liboschik:2016}. 

\subsection{Multivariate INGARCH model}
\label{sec: Multivariate Ingarch}

Since we have multivariate data, we also investigate multivariate versions of the INGARCH model. There have been various approaches in literature to expend the univariate INGARCH model to more dimensions. For example, bivariate models have been proposed by \cite{Liu:2012} and  extended by \cite{Cui:2018}. 

The authors in \cite{Fokianos:2020,Fokianos:2021} introduce and review the multivariate INGARCH model on the basis of a data generating process. Let $\bm{\lambda}_t=\mathbbm{E}[\bm{Y}_t|\SigA_t]$ where $\bm{\lambda}_t = (\lambda_{1t},\ldots,\lambda_{Kt})^T$ and $\SigA_t$ is the $\sigma$-field generated by $\left\{\bm{Y}_0,\ldots,\bm{Y}_t,\bm{\lambda}_0\right\}$. Then for each $k=1,\ldots,K$ we assume

\begin{gather}
Y_{kt} | \SigA_{t-1} \sim P(\lambda_{kt}), \\
\bm{\lambda}_t = \bm{d} + \bm{A}\bm{\lambda}_{t-1} + \bm{B}\bm{Y}_{t-1},
\label{eq:Mingarch 1}
\end{gather}

where $\bm{d}$ is a $K$-dimensional vector and $\bm{A},\bm{B}$ are $K\times K$ matrices. The elements of $\bm{d},\bm{A},\bm{B}$ are assumed to be positive such that $\bm{\lambda}_t > 0$. 

Based on this, a joint distribution is constructed using a copula structure and the process \cite{Fokianos:2020}

\begin{enumerate}
	\item Let $\bm{U}_l=(U_{1,l},\ldots,U_{K,l})$ for $l=1,\ldots,m$ be a sample from a $K$-dimensional copula $C(u_1,\ldots,u_K)$. Then by definition of a copula, $U_{i,l}$ follow marginally the uniform distribution on $(0,1)$ for $i=1,\ldots,K$ and $l=1,\ldots,m$. 
	\item Define the transformation $X_{i,l} = -\log (\frac{U_{i,l}}{\lambda_{i,0}})$. Then the marginal distribution of $X_{i,l}$ is exponential with parameter $\lambda_{i,0}$. 
	\item For $m$ large enough, define $Y_{i,0} = \max_{1\leq j \leq m}(\sum_{l=1}^j X_{i,l})\leq 1$. Then $\bm{Y}_0=(Y_{1,0},\ldots,Y_{K,0})$ is marginally a set of starting values of a Poisson process with parameter $\bm{\lambda}_0$. 
	\item Use model \ref{eq:Mingarch 1} to obtain $\bm{\lambda}$.
	\item Go back to step 1 to obtain $\bm{Y}_1$ and so on. 
\end{enumerate}

This construction of the joint distribution imposes the dependence among the components of the process $(\bm{Y}_t)_{t=1}^T$. This approach can be extended to other marginal count processes if they can be generated by continuous arrival times \cite{Fokianos:2020}. 

We can then define the multivariate INGARCH model as

\begin{gather}
\bm{Y}_t = \bm{N}_t(\bm{\lambda}_t), \\
\bm{\lambda}_t = \bm{d} + \bm{A}\bm{\lambda}_{t-1} + \bm{B}\bm{Y}_{t-1},
\label{eq:Mingarch 1 new}
\end{gather}

where $\left\{\bm{N}_t\right\}$ is a sequence of K-variate independent copula-Poisson processes that counts the number of events in $[0,\lambda_{1t}]\times,\ldots,\times[0,\lambda_{Kt}]$ \cite{Fokianos:2020}. 

Another approach is taken by \cite{Lee:2023}. Instead of constructing a joint distribution for the multivariate vector $\bm{Y}_t$, they fit a one-parameter exponential family conditional distribution to each component $Y_{kt}$

\begin{equation}
p_{k}(y|\nu) = \exp(\nu y - A_k(\nu))h_k(y), \hspace{0.5cm} y \in \mathbbm{N}_0,
\label{eq:Exponential Family}
\end{equation}

where $A_k$ and $h_k$ are known functions and $\nu$ is the natural parameter. Both $A_k$ and $B_k(\nu) = \frac{d A_k(\nu)}{d \nu}$ are strictly increasing \cite{Lee:2023}. The multivariate INGARCH model is then given for each $k=1,\ldots,K$ by

\begin{gather}
Y_{kt} | \SigA_{t-1} \sim p_k(y|\nu_{kt}), \\
\bm{\lambda}_t := \mathbbm{E}[\bm{Y}_t | \SigA_{t-1}] = f_{\theta}(\bm{\lambda}_{t-1},\bm{Y}_{t-1}),
\label{eq:Mingarch 2}
\end{gather}

where $\SigA_{t-1}$ is the $\sigma$-field generated by $\left\{\bm{Y}_{t-1},\bm{Y}_{t-2},\ldots,B_k(\nu_{kt})\right\}$ with $B_k(\nu_{kt})=\lambda_{kt}$, and $f_{\theta}$ is a non-negative function on $[0,\infty)^K \times \mathbb{N}_0^K$ \cite{Lee:2023}. So for each component $Y_{kt}$ a univariate INGARCH model is fit but the components are connected by the conditional mean process. A popular choice of $f_{\theta}$ results in a linear relationship. Take a $K$-dimensional vector $\bm{W}$ with positive entries and $K\times K$ matrices $\bm{A},\bm{B}$ with non-negative entries satisfying either \cite{Lee:2023}

\begin{equation}
sup_{\theta \in \Theta} (\sum_{j=1}^K(a_{ij}+b_{ij})) < 1, i=1,\ldots,K , 
\label{eq:Mingarch 2 function condition 1}
\end{equation}

for a compact set $\Theta \subseteq R^{K\times 2K^2}$ and $\theta=(\bm{W},\bm{A},\bm{B})$ or 

\begin{equation}
sup_{\theta \in \Theta}(\max_{1\leq j \leq K}(\sum_{i=1}^K a_{ij}) + \max_{1\leq j \leq K} \sum_{i=1}^K b_{ij}) < 1. 
\label{eq:Mingarch 2 function condition 2}
\end{equation}

Then model \ref{eq:Mingarch 2} becomes 

\begin{gather}
Y_{kt} | \SigA_{t-1} \sim p_k(y|\nu_{kt}), \\
\bm{\lambda}_t := \mathbbm{E}[\bm{Y}_t | \SigA_{t-1}] = \bm{W} + \bm{A}\bm{\lambda}_{t-1} + \bm{B}\bm{Y}_{t-1}. 
\label{eq:Mingarch 2 linear}
\end{gather}

%\section{Other Methods}
%\label{sec: Other methods}

\section{GARCH Models}
\label{sec: Garch Models}

INGARCH models are structurally derived from the generalised autoregressive conditional heteroscedasticity (GARCH) models, which themselves are generalisations of the autoregressive conditional heteroscedasticity (ARCH) model. ARCH models, which were first developed by Engle \textcite{Engle:1982} in an economic context, model the variance conditional on past values. Since we no longer assume our data to be integer valued, our time series has the form $\left\{Y_{kt}:t=1,\ldots T; Y_{kt} \in \mathbb{R}\right\}_f$. Denote again with $\SigA_{k,t}$ the information available at time $t$. Then the ARCH(1) model is given by \textcite{Engle:1982}

\begin{equation}
\begin{gathered}
Y_{kt} | \SigA_{k,t-1} \sim N(0,h_{kt}),\\
h_{kt} = a_0 + a_1 Y_{k,t-1}^2,
\label{eq:ARCH model}
\end{gathered}
\end{equation}
%
with $a_0\geq0$, $a_1>0$. 

The variance function can be generally formulated as $h_{kt} = h(Y_{k,t-1},\ldots,Y_{k,t-p},\bm{a})$, where $\bm{a}=(a_1,\ldots,a_p)^T\geq 0$ is the parameter vector with $a_p>0$, and $p \in \mathbbm{N}$ is the order of the ARCH process. 
The GARCH model generalises this approach by adding the past variances as another source of information. The GARCH(p,q) model for non-negative parameters $a_0>0$, $\bm{a}=(a_1,\ldots,a_q)^T\geq 0$ and $\bm{b}=(b_1,\ldots,b_p)^T\geq0$ with $p,q \in \mathbbm{N}$, $p\geq0, q>0$ is given by \textcite{Bollerslev:1986}

\begin{equation}
\begin{gathered}
Y_{kt} | \SigA_{k,t-1} \sim N(0,h_{kt}); \forall t \in \mathbbm{N}, \\
\mathbbm{V}[Y_{kt}|\SigA_{k,t-1}]=h_{kt} = a_0 + \sum_{i=1}^q a_i Y_{k,t-i}^2 + \sum_{j=1}^p b_j h_{k,t-j};\forall t \in \mathbbm{N}.
\label{eq:GARCH model}
\end{gathered}
\end{equation}
%
Other distributions than the normal distributions can be taken as well. 

\subsection{Parameter Estimation and Forecasting}
\label{sec: GARCH Parameter Estimation and Forecasting}

Estimation of the parameters can be done with maximum likelihood and an iterative algorithm. First, the model is rewritten and the logarithm of the likelihood function is taken. Second, after differentiation with respect to its variance and mean parameters, the Berndt, Hall Hall and Hausman algorithm \textcite{Berndt:1974}is used to obtain the maximum likelihood estimates. Further details and assumptions can be found in \textcite{Bollerslev:1986}. 

If one is interested in forecasting $Y_{kt}$, then the minimum mean squared one-step error forecast is $\mathbbm{E}[Y_{k,t+h}|\SigA_{k,t}]=0$ where $\SigA_{k,t}$ is the information available at time $t$. One should note, that the forecast is independent of the model parameters. If the conditional variance should be forecasted, the parameters are estimated and the known values are plugged in. For $h>1$, $h$-step predictions are computed recursively with plugging in the forecasts for $h-1,h-2,\ldots$ in the model \textcite{Zivot:2009}. 


\subsection{Testing for GARCH Models}
\label{sec: Testing for GARCH models}

To decide whether to use a GARCH model, one can test for volatility or the validity of GARCH models in general. In the original paper \textcite{Bollerslev:1986}, the author suggests a Lagrange multiplier test. Other popular tests include the Box–Pierce–Lung‐type portmanteau tests and residual‐based diagnostics \textcite{Hong:2017}. The authors in \textcite{Hong:2017} present further methods. 


\subsection{Applications}
\label{sec: Garch Applications}

The introduction of ARCH and subsequently GARCH models in the 1980s has been revolutionary. ARCH models have originally been introduced for modelling macroeconomic key figures such as inflation rates but since then have been used in a variety of fields. GARCH models generalised the ARCH model approach to allow the modelling of a more flexible lag structure \textcite{Bollerslev:1986}. They have found wide applications in finance mathematical problems, especially for the modelling of a changing variance and volatility in financial markets. They are often used to estimate volatility of various financial instruments. 

Since the ARCH and GARCH models are used to model and forecast volatility or the conditional variance, but not values, we will not use it in our application. In addition, the INGARCH model also accounts for the discrete nature of our data, which makes it the preferred choice. 


\section{Naive Random Walk}
\label{sec: Naive Random Walk}

The Naive Random Walk model is one of the simplest and most comprehensive forecasting models, which makes it a popular benchmark model. In addition, it is what is currently employed, so using it enables us to directly see if our models outperform the current model. It assumes that the 1-step difference between two values is i.i.d distributed with mean 0. Let $\left\{Y_{kt}:t=1,\ldots T; Y_{kt} \in \mathbb{N}_0\right\}_f$ be our univariate time series. Then the Naive Random Walk model is given as

\begin{equation}
Y_{k,t+1}= Y_{kt} + \epsilon_{kt}, 
\label{eq: Random Walk Model}
\end{equation}
 %
where $\epsilon_{kt} \sim WN(\sigma_k^2)$ is a white noise process with variance $\sigma_k^2 \in \mathbb{R}_+$. It can be shown easily, that the optimal 1-step ahead forecast with regard to the mean squared error (MSE) is given by

\begin{equation}
\hat{Y}_{k,t+1}= Y_{kt},
\label{eq: Random Walk Model Prediction}
\end{equation}
%
where $\hat{Y}_{k,t+1}$ is the predicted value at time $t+1$. In other words, the predicted value is the last known value.  


\section{Zero-Inflated Models}
\label{sec: Zim}

Since we encounter a large number of zeros, we also consider zero-inflated models. Zero inflation means that the proportion of observed zeros is bigger than that of the underlying distribution and hence would not be expected. The idea of zero-inflated models is to add a degenerated distribution with mass at zero to the probability mass function, which enables one to explain the large amount of zero values. The probability mass function of a $ZIP(\lambda,\omega)$ distribution for a random variable $Y$ is defined as \textcite{Zhu:2012}

\begin{equation}
\mathbb{P}(Y=y) = \omega \delta_{y,0}+ (1-\omega) \frac{\lambda^y \exp(-\lambda)}{y!}, \hspace{0.2cm} y \in \mathbb{N}_0.
\label{eq:ZIP Distribution}
\end{equation}
%
where $0 < \omega < 1$ is the zero-inflation parameter, $\lambda$ is the Poisson parameter and $\delta_{y,0}$ is the Kronecker delta for which $\delta_{y,0}=1$ if $y=0$ and $\delta_{y,0}=0$ else. This way our zeros can come from two different sources \textcite{Zhu:2012}. The first part of Equation (\ref{eq:ZIP Distribution}) $\delta_{y,0}$ is the degenerated point mass distribution.

Let $\SigA_{k,t-1}$ be the $\sigma$-field generated by $\left\{Y_{k,t-1},Y_{k,t-2},\ldots\right\}$. Assume, conditionally on $\SigA_{k,t-1}$, that $Y_{k,1},\ldots,Y_{k,T}$ are independent. Now we can define the Zero-Inflated Poisson (ZIP) INGARCH(p,q) as

\begin{equation}
\begin{gathered}
\label{eq:ZIP model}
Y_{kt} | \SigA_{k,t-1} \sim ZIP(\lambda_{kt},\omega_k); \forall t \in \mathbb{N}, \\
\mathbb{E}\left[Y_{kt} | \SigA_{k,t-1} \right] = \lambda_{kt} = \beta_0 + \sum_{i=1}^p\beta_i Y_{k,t-i} + \sum_{j=1}^q\alpha_j \lambda_{k,t-j},
\end{gathered}
\end{equation}
%
with $0<\omega_k<1$, $\beta_0>0$, $\beta_i\geq 0$, $\alpha_j \geq 0$ for $i=1,\ldots,p$, $j=1,\ldots,q$, $p\geq 1$, $q\geq 0$ \textcite{Zhu:2012}. If $\omega_k =0$ then we get the standard INGARCH(p,q) model discussed above. It can be shown that the conditional mean and variance are given by

\begin{equation}
\mathbb{E}[Y_{kt} | \SigA_{k,t-1}] = (1-\omega_k)\lambda_{kt}, \hspace{1cm} \mathbb{V}[Y_{kt} | \SigA_{k,t-1}] =(1-\omega_k)\lambda_{kt}(1+\omega_k \lambda_{kt}),
\label{eq:Conditional Mean and Variance ZIP}
\end{equation}
%
which implies $ \mathbb{V}[Y_{kt} | \SigA_{k,t-1}] > \mathbb{E}[Y_{kt} | \SigA_{k,t-1}]$ \textcite{Zhu:2012}. This means that Model (\ref{eq:ZIP model}) can handle overdispersion in our data. More details about zero-inflated models and especially the zero-inflated INGARCH(p,q) model can be found in \textcite{Zhu:2012}.

However, due to a lack of available R-packages for zero-inflated Poisson INGARCH models, we use a zero-inflated Poisson autoregressive model. We assume that we have discrete count data $\left\{Y_{kt}\right\}$ which is conditionally $ZIP(\lambda_{kt},\omega_{kt})$ distributed. For the parameters $\lambda_{kt}$ and $\omega_{kt}$, the ZIP autoregressive model is given by \textcite{Yang:2012}

\begin{equation}
\begin{gathered}
Y_{kt} | \SigA_{k,t-1} \sim ZIP(\lambda_{kt},\omega_{kt}); \forall t \in \mathbb{N}, \\
\log(\lambda_{kt}) = \sum_{i=1}^p\beta_i b^k_{t-1,i},\\ %\bm{B}^T_{k,t-1} \bm{\beta},\\
\log\left(\frac{\omega_{kt}}{1-\omega_{kt}}\right)=\sum_{i=1}^q\gamma_i z^k_{t-1,j},%\bm{Z}_{k,t-1}^T\bm{\gamma},
\label{eq:ZIP Autoregressive model}
\end{gathered}
\end{equation}
%
where $\bm{\beta} = (\beta_1,\ldots,\beta_p)^T$ and $\bm{\gamma}=(\gamma_1,\ldots,\gamma_q)^T$ are the parameters to be estimated and the vectors $\bm{B}^k_t=(b^k_{tj})$ and $\bm{Z}^k_{t}=(z^k_{tj})$ are the explanatory covariates. In Model (\ref{eq:ZIP Autoregressive model}) a logit link function has been used although, it can be replaced with other link functions like the probit or log link. 

In our case we regress on the past values of our time series. In that case Model (\ref{eq:ZIP Autoregressive model}) becomes

\begin{equation}
\begin{gathered}
\log(\lambda_{kt}) = \beta_1 + \beta_2 Y_{k,t-1},\\%(1,Y_{k,t-1}) \bm{\beta},\\
\log\left(\frac{\omega_{kt}}{1-\omega_{kt}}\right)= 1 \cdot \gamma.
\label{eq:ZIP Autoregressive model timeseries}
\end{gathered}
\end{equation}

%Hence we have $\bm{B}_{t} = (1,Y_{k,t})$ and $\bm{Z}_{t} = 1$. 

\subsection{Parameter Estimation and Forecasting}
\label{sec: ZIM Parameter Estimation and Forecasting}

Parameter estimation for Model (\ref{eq:ZIP Autoregressive model timeseries}) is done with the maximum partial likelihood estimate. However, since there exists no closed form solution, iterative algorithms like the Expectation-Maximisation (EM), Newton-Raphson (NR) or Fisher Scoring (FS) can be used.  Further details can be found in \textcite{Yang:2012}.

The one step ahead predictor is again given by the conditional expectation $\mathbb{E}[Y_{kt} | \SigA_{k,t-1}] = (1-\omega_{kt})\lambda_{kt} $ with the estimated coefficients plugged in. 

\section{Log-Linear Models}
\label{sec: Log-Linear Models}

As mentioned in Section \ref{sec:Ingarch Motivation}, we also investigate log-linear models. These models are structurally very similar to the normal INGARCH(p,q) model, only with a logarithmic link function. Under the same assumptions as for Model (\ref{eq:Ingarch model}), they have the form 

\begin{equation}
\begin{gathered}
Y_{kt} | \SigA_{k,t-1} \sim P(\lambda_{kt}); \forall t \in \mathbb{N}, \\
\nu_{kt}= \log(\lambda_{kt}) = \beta_0 + \sum_{i=1}^p\beta_i \log(Y_{k,t-i}+1) + \sum_{j=1}^q\alpha_j \nu_{k,t-j}.
\label{eq:Log-Linear model}
\end{gathered}
\end{equation}
%
The past values get transformed by $h(x)=\log(x+1)$ to get them on the same scale as $\nu_{kt}$ and avoid zero values in the logarithm \textcite{Liboschik:2016,Fokianos:2011}. We consider the log-linear model because it provides solutions to at least two drawbacks from the INGARCH(p,q) model. First, as a result of the definition of the parameter space, Equation (\ref{eq:Ingarch parameter space}), we have $0 < \sum_{i=1}^p\beta_i + \sum_{j=1}^q\alpha_j < 1$ and hence it follows for $h\in \mathbb{N}$ that $Cov(Y_{k,t+h},Y_{kt})>0$. Second, when we include covariates, they can only have a positive regression term because otherwise the mean $\lambda_{kt}$ becomes negative \textcite{Fokianos:2011}. However, in the log-linear case we can extend this to

\begin{equation}
\begin{gathered}
Y_{kt} | \SigA_{k,t-1} \sim P(\lambda_{kt}); \forall t \in \mathbb{N}, \\
\nu_{kt}= \log(\lambda_{kt}) = \beta_0 + \sum_{i=1}^p\beta_i \log(Y_{k,t-i}+1) + \sum_{j=1}^q\alpha_j \nu_{k,t-j} + \bm{\eta}^T\textbf{X}_{kt}.
\label{eq:Log-Linear model external factors}
\end{gathered}
\end{equation}
%
with $\bm{\eta} \in \mathbb{R}^r$. Additionally, because of the updated definition of the parameter space, Equation (\ref{eq:Parameter Space log-linear}), it also allows for negative autocorrelation \textcite{Liboschik:2016}. 

\subsection{Parameter Estimation and Forecasting}
\label{sec: Log-Linear Parameter Estimation and Forecasting}

The parameter estimation for the log-linear model is done similarly to the INGARCH model in Section \ref{sec: Estimation of the Ingarch Model}. Only the parameter space $\Theta$ is different

\begin{equation}
\Theta = \left\{ \bm{\theta} \in \mathbb{R}^{p+q+r+1}: \abs{\beta_1},\ldots,\abs{\beta_p},\abs{\alpha_1},\ldots,\abs{\alpha_q} < 1, \abs{\sum_{i=1}^p\beta_i + \sum_{j=1}^q\alpha_j } < 1 \right\}.
\label{eq:Parameter Space log-linear}
\end{equation}
%
Just like parameter estimation, forecasting is also performed in the same way as the INGARCH model. The optimal one-step ahead prediction with regards to the mean squared error is given by the conditional expectation $\lambda_{k,t+1}=\mathbb{E}\left[Y_{k,t+1} | \SigA_{kt} \right]$. The h-step ahead predictions for $h>1$ are calculated iteratively again \textcite{Liboschik:2016}. 

Log-Linear models are further discussed in \textcite{Fokianos:2011,Woodard:2011,Douc:2013}.


%\section{Vector Generalised Additive Models}
%\label{sec:Vgam}
%
%Because we work with multivariate count data, we also look at vector generalised additive models (VGAMs) which extend generalised additive models (GAMs) to higher dimensions. GAMs allow us to reveal and model non-linear relationship in our data, as opposed to linear models or generalised linear models \textcite{Yee:1996}. Let $y$ be a univariate response with a distribution in the exponential family and mean $\mu$. Further take a p-dimensional covariate vector and $\bm{x}=(x_1,\ldots,x_p)^T$. Then the generalised additive model (GAM) is given by
%
%\begin{equation}
%g(\mu) = \nu(\bm{x}) = \beta_0 + f_1(x_1) + \ldots f_p(x_p),
%\label{eq:Gam}
%\end{equation}
%
%with $f_j$ being arbitrary smooth functions \textcite{Yee:1996}.
%To extend this model to the multivariate case, we replace the functions $f_j$ with vector functions. Let $\bm{f}_k(Y_{kt}) = (f_{(1)k}(Y_{kt}),\ldots,f_{(M)k}(Y_{kt}))^T$ with $M \in \mathbbm{N}$ be an arbitrary smooth vector function. Then the vector generalised additive model is given by
%
%\begin{equation}
%\mathbb{E}[\bm{Y}_t] = \bm{\beta}_0 + \sum_{k=1}^K\bm{f}_k(Y_{k,t-1}),
%\label{eq:Vgam}
%\end{equation}
%
%where $\mathbb{E}[\bm{Y}_t] = (\mathbb{E}[Y_{1t}],\ldots,\mathbb{E}[Y_{Kt}])^T$ \textcite{Yee:1996}. Further theoretical background about VGAMs are given in \textcite{Yee:1996,Yee:2015,Wood:2004}.

\section{AR Models}
\label{sec: Ar Models}

Autoregressive models of order p (AR(p))  are one of the most simple time series models, which makes them very popular. For a stationary process $\left\{x_t\right\}\ $, $\bm{a}=(a_1,\ldots,a_p)^T \in \mathbbm{R}^p$ and a white noise process $\epsilon_{t} \sim WN(\sigma^2)$, called innovations, they are defined as 

\begin{equation}
x_{t} = a_1x_{t-1} + \ldots + a_p x_{t-p} + \epsilon_{kt},
\label{eq: Ar model}
\end{equation}
%
where $\left\{Y_{kt}:t=1,\ldots T; Y_{kt} \in \mathbb{R}\right\}_f$ is again our univariate time series. As with the ARCH and GARCH models, we no longer assume that the data is constrained to integer values. The multivariate version, also called vectorised autoregressive model (VAR(p)), is defined as 

\begin{equation}
\bm{Y}_{t} = \bm{a_1 Y_{t-1}}+ \ldots + \bm{a_p Y_{t-p}} + \bm{\epsilon}_{t},
\label{eq: Var Model}
\end{equation}
%
where $\bm{a}_j \in \mathbbm{R}^{k \times k}$ are the coefficient matrices, $\bm{\epsilon}_t \sim WN(\bm{\Sigma})$ is a white noise process with covariance matrix $\bm{\Sigma}$ and $\bm{Y}_t = (Y_{1t},\ldots,Y_{Kt})^T$.% with $\Sigma \in \mathbbm{R}^{k \times k}$. 

\subsection{Parameter Estimation and Forecasting}
\label{sec: AR Estimation and Forecasting}

The simplicity of AR models makes parameter estimation and forecasting easy. There are various ways to estimate the parameters in Model (\ref{eq: Var Model}) such as the Yule-Walker equations, the ordinary least squares (OLS) estimator and if the innovations $(\bm{\epsilon}_t)$ are multivariate normal distributed, then the maximum likelihood estimator can be used as well. Further properties and comparison of their estimators can be found in \textcite{Scherrer:2021}. 

Like parameter estimation, forecasting is also simple in the AR model. We use the mean squared error as an error measure, only consider affine forecasts, assume w.l.o.g. that  has expectation 0 and using $m\geq p,m \in \mathbbm{N}$ past values, we get that our forecast has the form 

\begin{equation}
\hat{\bm{Y}}_{t+h} = \bm{c}_1\bm{Y}_{t} + \ldots + \bm{c}_m\bm{Y}_{t-m+1},
\label{eq:Forecasting general}
\end{equation}
%
with coefficients $\bm{c}_i \in \mathbbm{R}^{k \times k}$. It turns out that the optimal one-step ahead prediction is simply \textcite{Scherrer:2021}

\begin{equation}
\hat{\bm{Y}}_{t+1} = \bm{a}_1\bm{Y}_{t} + \ldots + \bm{a}_p\bm{Y}_{t-p+1}.
\label{eq:AR 1step Forecasting}
\end{equation}
%
For $h>1$, one simply continues recursively, using $\hat{\bm{Y}}_{t+1}$, $\hat{\bm{Y}}_{t+2},\ldots$ \textcite{Scherrer:2021}. Since we normally do not know the exact coefficients $\bm{a}_i$, their estimations $\hat{\bm{a}}_i$ are plugged into Equation (\ref{eq:AR 1step Forecasting}). 

\subsection{Testing for AR Models}
\label{sec: Testing for ar models}

To test whether or not a time series follows an AR(p) process, the estimates of the white noise process $\hat{\epsilon}_t$ can be used. These estimates should follow a white noise process and hence should show no signs of serial correlation. Popular tests are the Portmanteau and the Breusch-Godfrey Test \textcite{Scherrer:2021}. 

The Portmanteau Test tests the null hypothesis $H_0: \mathbbm{E}[\bm{\epsilon}_t \bm{\epsilon}_{t-m}^T]=0$, i.e. if the estimated innovations are uncorrelated. Under the assumption that $\left\{\bm{Y}_t\right\}$ is an AR(p) process and an AR(p) model has been fit, the used test statistic converges against a chi-squared distribution \textcite{Scherrer:2021}. 

The Breusch-Godfrey Test tests if the coefficients $(b_1,\ldots,b_h)$ in the model 

\begin{equation}
\bm{\epsilon}_t = d_1\bm{\epsilon}_{t-1} +\ldots d_h \bm{\epsilon}_{t-h} + \bm{\eta}_t,
\label{eq:Breusch-Godfrey Test model}
\end{equation}
%
are zero, i.e. if process $(\bm{\epsilon}_t)$ follows an AR(h) structure or not. Under the null hypothesis the test statistic follows a chi-squared distribution again \textcite{Scherrer:2021}. 

\section{INAR(p) Models}
\label{sec: Inar Models}

Integer valued autoregressive models of order p (INAR(p)) are another option to handle univariate count data. To define them, we first need to define the generalised thinning operator. Take an integer valued, non-negative random variable $X$ and $\alpha \in [0,1]$. Further, take a sequence of i.i.d. integer valued, non-negative random variables $(Z_i)_{i=1}^X$ with finite mean $\alpha$ and variance $\sigma^2<\infty $ which are independent of $X$. Then the generalised thinning operator $\circ$ is defined as

\begin{equation}
\alpha \circ X = \sum_{i=1}^X Z_i .
\label{eq:Thinning operator}
\end{equation}
%
The sequence  $\left\{Z_i \right\}_{i=1}^X$ is called the counting series of $X$ \textcite{Silva:2005}. 

We can then define the INAR(p) model for a positive integer valued time series $\left\{X_t \right\}$ as

\begin{equation}
X_t = \alpha_1 \circ X_{t-1} + \alpha_2 \circ X_{t-2} + \ldots + \alpha_p X_{t-p} +\epsilon_t ,
\label{eq:Inar(p) model}
\end{equation}
%
where

\begin{enumerate}
	\item $(\epsilon_t)$ is a sequence of integer valued i.i.d. random variables, called innovations, with finite first, second and third moment, 
	\item $\alpha_i \circ X_{t-i}$ for $i= 1,\ldots,p$ and $(Z_j)$ for $j=1,\ldots,X_{t-i}$ are mutually independent, independent of $(\epsilon_t)$ and it holds $\mathbb{E}[Z_{i,j}]=\alpha_i$, as well as $\mathbb{V}[Z_{i,j}] = \sigma_i^2$ and $\mathbb{E}[Z_{i,j}^3] = \gamma_i$,
	\item $\alpha_i \in (0,1]$ for $i=1,\ldots,p-1$ and $0 < \alpha_p < 1$,
	\item $\sum_{j=1}^p \alpha_j < 1$ \textcite{Silva:2005}. 
\end{enumerate}


The last condition ensures the existence and stationary of the process. 

Let $\left\{Y_{kt}:t=1,\ldots T; Y_{kt} \in \mathbb{N}_0\right\}_f$ be again the univariate time series for category $k$ for $k=1,\ldots,K$ and fridge $f$. Then the INAR(p) model is given by

\begin{equation}
Y_{kt} = \alpha_1 \circ Y_{k,t-1} + \alpha_2 \circ Y_{k,t-2} + \ldots + \alpha_p Y_{k,t-p} +\epsilon_{kt}.
\label{eq:Inar(p) model ts}
\end{equation}
%
For simplicity, we will consider INAR(1) models, although the optimal choice of the lag is something that could be further investigated. 

\subsection{Distributional Assumptions}
\label{sec: Inar Distributional assumptions}

While we will mainly assume that the innovations $(\epsilon_t)$ follow a Poisson distribution, they can also follow other distributions. One interesting option is, that one can choose a zero-inflated distribution. This could make the model adequate for our data. 


\subsection{Parameter Estimation and Forecasting}
\label{sec: Inar Parameter Estimation and Forecasting}

Parameter estimation can be done in several ways. Possible methods are: moment based estimators (MM), regression based or conditional least squares (CLS) and maximum likelhood (ML) based estimators. Especially for the Poisson model, those methods have been studied in detail in literature \textcite{Silva:2005}. 

The authors in \textcite{Silva:2005} present two types of forecasting methods for INAR(1) models. The first approach is a classical method for performing predictions in a time series context and makes use of the conditional expectation. It was obtained by \textcite{Bre:1993} and \textcite{Freeland:2004}. Assuming that $(\epsilon_t) \sim_{i.i.d} P(\lambda)$, the $h$-step ahead predictor, for $h\in \mathbbm{N}$, based on $n$ past observations $\bm{Y}_k=(Y_{k1},\ldots,Y_{kn})$ is given by

\begin{equation}
\hat{Y}_{k,n+h} = \mathbb{E}[Y_{k,n+h} | \bm{Y}_k] = \alpha^h \left[Y_{kn}- \frac{\lambda}{1-\alpha} \right] + \frac{\lambda}{1-\alpha}.
\label{eq:Forecasting Classic}
\end{equation}


However, this forecast hardly ever produces integer values. One option to counter this problem, is to take the value which minimises the absolute expected error $\mathbbm{E}[\abs{Y_{k,n+h} - \hat{Y}_{k,n+h}}|Y_{k,n}]$ instead of the MSE. This turns out to be the median $\hat{m}_{n+h}$ of the h-step ahead conditional distribution of $Y_{k,n+h}|Y_{k,n}$ \textcite{Silva:2005,Freeland:2004}. Another option is a bayesian approach presented in \textcite{Silva:2005}. It is based on the assumption that both, the future prediction $Y_{k,n+h}$ and the vector of unknown parameters $\bm{\theta}=(\alpha,\lambda)$ are random \textcite{Silva:2005}. Since the complexity posterior probability density function makes it difficult to work with it directly, a sampling algorithm can be deployed for estimation. The details are again given in \textcite{Silva:2005}. The estimator for the conditional expectation is then given by

\begin{equation}
\hat{Y}_{k,n+h}= Y_{kn}\left(\frac{1}{m} \sum_{i=1}^m\alpha_i^m\right) + \left(\frac{1}{m} \sum_{i=1}^m \frac{1-\alpha_i^h}{1-\alpha_i}\lambda_i\right),
\label{eq:Forecasting Bayesian}
\end{equation}
%
where $m$ is the sampling size and the pairs $(\alpha_i,\lambda_i)$ for $i=1,\ldots,m$ are the sampled parameters. 

\subsection{Testing for INAR(1) Models}
\label{sec:Testing for INAR(1) Models}

To test the adequacy of the INAR(1) model, there are again various options. 

Parametric re-sampling is a popular method. The idea is to generate data with the help of the fitted model, construct the empirical distribution of the functional of interest and check if the original sample is a reasonable point within that empirical distribution \textcite{Silva:2005}. 

Residual based methods are based on the Pearson residuals defined by 

\begin{equation}
r_{kt} = \frac{Y_{kt}-\mathbbm{E}[Y_{kt}|Y_{k,t-1}]}{\mathbbm{V}[Y_{kt}|Y_{k,t-1}]^{1/2}},
\label{eq:Pearson residuals}
\end{equation}
%
where estimated quantities are plugged in. If the model is specified correctly, the residuals should have mean zero, variance one and no significant serial correlation \textcite{Silva:2005}. 

Another option is based on predictive distributions where an adjusted probability integral transform (PIT) is used. Further details can be found in \textcite{Silva:2005}. 

\subsection{Difference to AR(p) Models}
\label{sec: Difference to AR models}

%Using other defintion of INAR(p) model. 
%While structurally the AR(P) and then INAR(P) model look similar they have different properties. The only exception is for $p=1$. The authors in \textcite{McKenzie:1986} and \textcite{Alzaid:1988} noted that the aside from their structural similarity, those two processes also show the same autocorrelation and regression behaviour \textcite{Alzaid:1990}. However, for $p>1$ the similarities end at the structure. For an INAR(p) process, the components $\alpha_i \circ Y_{t,k}$ of $Y_{t,k}$ have a mutual dependence structure for $i=1,2,\ldots,p$ which induces a moving-average structure. In the AR(p) process they are only connected by the presence of $Y_{t,k}$. Because of this additional dependence, it can be shown that the autocorrelation behaves more like a standard ARMA(p,p-1) process \textcite{Alzaid:1990}. 

Depending on the definition of the INAR(p) model, the degree of similarity varies. If one follows the definition of \textcite{Guan:1991}, which is the one given in Equation (\ref{eq:Inar(p) model}), then the autocorrelation function follows that of an AR(p) process \textcite{Oliveira:2005}. However, the authors in \textcite{Alzaid:1990} propose a different definition. In their work, given $Y_{tk}=y_{tk}$, the conditional distribution of $(\alpha_1 \circ Y_{tk}, \ldots, \alpha_p \circ Y_{tk})$ is multinomial with parameters $( \alpha_1,\ldots,\alpha_p,y_{tk})$ and is independent of the history of the process. Under those assumptions, the components $\alpha_i \circ Y_{t,k}$ of $Y_{t,k}$ for $i=1,2,\ldots,p$  have a stronger mutual dependence structure than the corresponding AR(p) process and induce a moving-average structure \textcite{Alzaid:1990}. Because of this additional dependence, it can be shown that the autocorrelation behaves more like a standard ARMA(p,p-1) process \textcite{Alzaid:1990}. 