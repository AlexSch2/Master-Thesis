\section{Motivation}
\label{sec:Motivation}

\section{Data Description}
\label{sec: Data Description}

The foundation and starting step of our analysis is the structure of our data. Roughly speaking, we have multiple multivariate count data timeseries. Each timeseries represents a fridge and its values represent the number of items sold. The items sold, are categorised in 4 main categories and each main category is further split up in various sub categories. For a fridge $f$ denote this timeseries with 

\begin{equation}
\left\{\bm{Y}_t:t\in \mathbb{N}, \bm{Y}_t \in \mathbb{N}_0^K \right\}_f
\label{eq:Timeseries definition}
\end{equation}

where $K$ stands for the number of categories and $\mathbb{N}_0^K := \underbrace{\mathbb{N}_0 \times \mathbb{N}_0}_{K-times}$. 
The data is measured weekly and hence our points in time are equidistant. A special feature of our data is the amount of 0 and NA values. How they are handled is explained in later sections. 