\begin{otherlanguage}{german}

In dieser Diplomarbeit werden zwei Ansätze für multivariate Zähldaten-Zeitreihen mit einer überproportionalen Anzahl von Nullern analysiert und verglichen. Der erste Ansatz gehört zu der Klasse der verallgemeinerten linearen Modelle (GLM). Dabei wird ein ganzzahliges verallgemeinertes autoregressives Model mit bedingter Heteroskedastizität der Ordnung (p,q) (INGARCH(p,q) Model) für jede Dimension gefitted. Der zweite Ansatz basiert auf der Analyse von Kompositionsdaten (CoDA) und nutzt die relative Struktur der Daten, um daraus ein vektorisiertes autoregressives Model (VAR) zu erstellen. Darüber hinaus betrachten wir auch alternative Optionen wie Zero-Inflation-Modelle (ZIM) und ganzzahlige autoregressive Modelle (INAR). Wir beschreiben den mathematischen Hintergrund des INGARCH(p,q)- und des CoDA-Ansatzes, untersuchen verschiedene Parametereinstellungen, vergleichen Tuning-Optionen und testen die Modelle mit realen Daten. Anschließend führen wir ein Fehlermaß zum Vergleich ein und verwenden es, um die Güte der Modelle bei verschiedenen Zeitreihen zu vergleichen. Schließlich stellen wir ein Benutzerhandbuch für unsere Analyse in der Statistiksoftware R zur Verfügung und präsentieren die verwendeten Pakete und Funktionen. \newline

\textbf{Schlagworte:} Compositional Data Analysis, General Linear Models, INGARCH, Multivariate Count Data Time Series, R

\end{otherlanguage}